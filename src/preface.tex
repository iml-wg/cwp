To outline the challenges in computing that high-energy physics will face over the next years and strategies to approach them, the HEP software foundation has organised a Community White Paper (CWP)~\cite{mainCWP}. In addition to the main document, several more detailed documents were worked out by different working groups. The present document focusses on the topic of machine learning.
The goals are to define the tasks at the energy and intensity frontier that can be addressed during the next decade by research and development of machine learning applications. % In this context, the interaction of the high-energy physics and data science communities is discussed, prioritized areas of machine learning R\&D are proposed, machine learning software and hardware platforms and resources are analyzed and a roadmap of future sustainable development is presented.

Machine learning in particle physics is evolving fast, while the contents of this community white paper were mainly compiled during community meetings in spring 2017 that took place at several workshops on machine learning in high-energy physics: S2I2 and \cite{DSatHEP2017,IML2017,ACAT2017,HSF2017}.
The contents of this document thus reflect the state of the art at these events and does not attempt to take later developments into account.
