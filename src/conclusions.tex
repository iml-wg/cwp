% Conclusions
% too generic, and too specific at the same time. applies to all CWPs, but doesn't cover nearly all applications from the earlier parts of the CWP: As particle physics moves into the post-Higgs boson discovery era, the physics drivers of the High-Luminosity Large Hadron Collider and future neutrino experiments will require increasingly more powerful identification and reconstruction algorithms to extract rare signals from copious and challenging backgrounds.

Machine learning algorithms are already state of the art in many areas of particle physics and will likely be called on to take on a greater role in solving upcoming data analysis and event reconstruction challenges.
In this document we have outlined the promising areas of research and development applications of machine learning in particle physics and focused on addressing the most important science drivers.
We identified the need for greater collaboration with external communities in machine learning and a need to train the particle physics community in machine learning.
We provided an example roadmap for the acceptance and implementation of machine learning applications into the workflows of particle physics experiments.
