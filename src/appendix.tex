\subsection{Matrix Element Methods}
\label{subsec:MEM}

The ME method is based on \emph{ab initio} calculations of the probability density function $\mathcal{P}$ of an event with observed final-state particle momenta ${\bf x}$ to be due to a physics process $\xi$ with theory parameters $\boldsymbol\alpha$.
One can compute $\mathcal{P}_{\xi}({\bf x}|{\boldsymbol\alpha})$ by means of the factorization theorem from the corresponding partonic cross-sections of the hard-scattering process involving parton momenta ${\bf y}$ and is given by
\begin{equation}
 \mathcal{P}_{\xi}({\bf x}|{\boldsymbol\alpha}) = \frac{1}{\sigma^{\rm fiducial}_{\xi}(\boldsymbol\alpha)} \int d\Phi ({\bf y}_{\rm final}) \; dx_1 \; dx_2~\frac{f(x_1)f(x_2)}{2s x_1 x_2} \; |\mathcal{M}_{\xi}({\bf y}|\boldsymbol\alpha)|^2 \; \delta^{4}({\bf y}_{\rm initial}-{\bf y}_{\rm final}) \; W({\bf x}, {\bf y})
 \label{eqn:MEProb}
\end{equation}
where and $x_i$ and ${\bf y}_{{\rm initial}}$ are related by $y_{{\rm initial},i}\equiv \frac{\sqrt{s}}{2}(x_i,0,0,\pm x_i)$, $f(x_i)$ are the parton distribution functions, $\sqrt{s}$ is the collider center-of-mass energy, $\sigma^{\textrm{ fiducial}}_{\xi}(\boldsymbol\alpha)$ is the total cross section for the process $\xi$ (with $\boldsymbol\alpha$) times the detector acceptance, $d\Phi({\bf y})$ is the phase space density factor, $\mathcal{M}_{\xi}({\bf y}|\boldsymbol\alpha)$ is the matrix element (typically at leading-order (LO)), and $W({\bf x}, {\bf y})$ is the probability density (aka ``transfer function'') that a selected event ${\bf y}$ ends up as a measured event ${\bf x}$.
One can use calculations of Eq.~\ref{eqn:MEProb} in a number of ways (e.g. likelihood functions) to search for new phenomena at particle colliders.



%%%


As stated in Sect.~\ref{sec:applications-MEM}, the ME method has three notable features: it (1) does not require training data being an \emph{ab initio} calculation of event probabilities, (2) incorporates all available kinematic information of a hypothesized process, including all correlations, and (3) has a clear physical meaning in terms of the transition probabilities within the framework of quantum field theory.

In reference to point (1), the matrix element $\mathcal{M}_{\xi}({\bf y}|\boldsymbol\alpha)$ in the method involves all partons in the $n\rightarrow m$ process, so when the 4-momentum of particles are not completely measured experimentally (e.g. neutrinos), one must integrate over the missing information which increases the dimensionality of the integration.
In reference to point (2), a clever technique to re-map the phase space in order to reduce the sharpness of integrate in that space in an automated way ({\sf MADWEIGHT}~\cite{Artoisenet:2010cn}) is often used in conjunction with a matrix element calculation package ({\sf MADGRAPH\_aMC\@NLO}~\cite{Alwall:2014hca}).
In practice, evaluation of definite integrals by the ME approach invokes techniques such as importance sampling (see {\sf VEGAS}~\cite{PETERLEPAGE1978192,Ohl:1998jn} and {\sf FOAM}~\cite{JADACH200355}) or recursive stratified sampling (see MISER~\cite{Press:1989vk}) Monte Carlo integration.
Acceleration of some of these techniques on modern computing architectures has been achieved, for example concurrent phase space sampling in VEGAS on GPUs.
