Machine learning has been applied to several problems in particle physics research, beginning with applications to high-level physics analysis in the 1990s and 2000s, followed by an explosion of applications in particle and event identification and reconstruction in the 2010s.
In this document we discuss promising future research and development areas for machine learning in particle physics. We detail a roadmap for their implementation, software and hardware resource requirements, collaborative initiatives with the data science community, academia and industry, and training the particle physics community in data science.
The main objective of the document is to connect and motivate these areas of research and development with the physics drivers of the High-Luminosity Large Hadron Collider and future neutrino experiments and identify the resource needs for their implementation.
Additionally we identify areas where collaboration with external communities will be of great benefit.
