In order to address the communication barrier and to speak the same language, the HEP community should be trained in ML concepts and terminology as part of a standard curriculum. The training should focus on well-maintained and well-documented software packages. It should provide lectures on general ML concepts and hands-on tutorials on specific tools based on concrete examples.\\

Being able to apply machine learning to practical HEP problems requires the understanding of basic ML concepts and algorithms. For this, regular data science lecture series and seminars, like~\cite{mlhep}, are very useful. At the university level, courses dedicated to machine learning applications in physics research are an excellent way to train undergraduate and graduate students. For example, the ``Deep Learning in Physics Research'' course with 60 participants consisting of 12 lectures and exercises which are performed on 20 GPUs of the VISPA internet platform~\cite{vispa}.\\ %Such resources should be made more centrally available.

Experiments currently have training activities for newcomers that focus on analysis software and introduction to domain knowledge~\cite{2016chep.confE.334B}. Machine learning should next be incorporated into the incoming collaborators training efforts of the experiments.
As discussed in the Community White Paper report on Careers and Training~\cite{HSF-CWP-2017-02}, ensuring the development and availability of resources for knowledge transfer is likewise essential to ML.
%---------------------------------------------------------------
% THE TRAINING SECTION HAS MOVED:
%  https://www.sharelatex.com/project/595500273c5204ff35dfdcf9
%--------------------------------------------------------------